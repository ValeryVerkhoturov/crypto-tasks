% !TeX program = xelatex
% !TeX encoding = UTF-8
% !TeX root = verkhoturov.tex

\begin{titlepage}
	
	\pagestyle{empty}
	\setlength\parindent{0pt}
	\newcommand{\blankDate}[2]{\mbox{\uline{<<\makebox[.7cm]{#1}>>~\makebox[2cm]{#2}~\the\year{}~г.}}} % {день}{месяц}
	\newcommand\blankLine[2]{$\underset{\text{#1}}{\text{\uline{#2}}}$}
	\begin{center}
		\includegraphics[width=2.5cm]{MIREA_Gerb_Black} \par
		МИНОБРНАУКИ РОССИИ \par 
		Федеральное государственное бюджетное образовательное учреждение высшего образования \par
		\textbf{<<МИРЭА~--- Российский технологический университет>>} \par
		\textbf{\fontsize{16pt}{16pt}\selectfont РТУ МИРЭА} \par
		\blankLine{(наименование института, филиала)}{Институт кибербезопасности и цифровых технологий} \par
		\blankLine{(наименование кафедры)}{Кафедра КБ-14 <<Цифровые технологии обработки данных>>} \par
		\vspace*{1cm}
		{\fontsize{16pt}{16pt}\selectfont
			\textbf{Практическая работа}} \par
		по дисциплине \blankLine{(наименование дисциплины)}{Криптографическая защита информации}
	\end{center}
	\vspace*{1cm}
	\begin{flushright}
		Выполнил: \par
		БСБО-05-20 \par
		Верхотуров В. С.
	\end{flushright}
	
	\begin{center}
		\vfill Москва~--- \the\year{}~г.
	\end{center}
\end{titlepage}
\addtocounter{page}{3}


\tableofcontents

\section{Шифр многоалфавитной замены Вижинера}

\subsection{Задание}

Для повышения стойкости шифра используют многоалфавитные замены,
в которых для замены символов исходного текста используются символы
нескольких алфавитов.

Одной из разновидностей такого метода является схема шифрования
Вижинера. Шифр очень устойчивый к вскрытию. Таблица Вижинера
представляет собой квадратную матрицу с n2 элементами, где n – число
символов используемого алфавита. Каждая строка получена циклическим
сдвигом алфавита на один символ

При шифровании сообщения его выписывают в строку, а под ним
буквенный ключ. Если ключ оказался короче сообщения, то его циклически
повторяют. Шифровку получают, находя символ в колонке таблицы по букве
текста и строке, соответствующей букве ключа.

Например:

Сообщение П Р И Е З Ж А Ю Ш Е С Т О Г О

Ключ А Г А В А А Г А В А А Г А В А

Шифровка П Н И Г З Ж Ю Ю ЮА Е О Т М

Предыдущие шифры называются монограммными, так как шифрование ведется по одной букве. Шифрование по 2 букве называются биграммными.

\subsection{Выполнение практической}

Результат практической: \url{https://crypto-tasks.vercel.app/task1}.

Репозиторий \url{https://github.com/ValeryVerkhoturov/crypto-tasks}.

\begin{lstlisting}[caption={Шифр многоалфавитной замены Вижинера}, label=lst:1]
export class VigenereCipher {
	private alphabet: string = 'АБВГДЕЁЖЗИЙКЛМНОПРСТУФХЦЧШЩЪЫЬЭЮЯ';
	private mod: number = this.alphabet.length;
	
	encrypt(text: string, key: string): string {
		return this.processText(text, key, 'encrypt');
	}
	
	decrypt(text: string, key: string): string {
		return this.processText(text, key, 'decrypt');
	}
	
	private processText(text: string, key: string, mode: 'encrypt' | 'decrypt'): string {
		let processedText = '';
		let keyIndex = 0;
		
		text = text.toUpperCase();
		key = key.toUpperCase();
		
		for (let i = 0; i < text.length; i++) {
			const char = text[i]
			if (this.alphabet.includes(char)) {
				const charIndex = this.alphabet.indexOf(char);
				const keyChar = key[keyIndex % key.length];
				const keyCharIndex = this.alphabet.indexOf(keyChar);
				
				if (mode === 'encrypt') {
					processedText += this.alphabet[(charIndex + keyCharIndex) % this.mod];
				} else {
					let decodeIndex = (charIndex - keyCharIndex) % this.mod;
					if (decodeIndex < 0) {
						decodeIndex += this.mod;
					}
					processedText += this.alphabet[decodeIndex];
				}
				
				keyIndex++;
			} else {
				processedText += char;
			}
		}
		
		return processedText;
	}
}

\end{lstlisting}

\section{Магический квадрат}

\subsection{Задание}

Магический квадрат - это квадратная таблица с вписанными в клетки последовательными натуральными числами от 1, которые дают в сумме по каждому столбцу, каждой строке и каждой диагонали одно и тоже число.

Чтобы зашифровать открытый текст с помощью такого квадрата, нужно пронумеровать все буквы в фразе по порядку и вставить их в квадрат вместо соответствующих цифр.

\subsection{Выполнение практической}

Результат практической: \url{https://crypto-tasks.vercel.app/task2}.

Репозиторий \url{https://github.com/ValeryVerkhoturov/crypto-tasks}.

\begin{lstlisting}[caption={Шифр Магический квадрат}, label=lst:2]
export class MagicSquareCipher {
	private magicSquare: number[][]
	private magicSquareDimensions: number
	
	constructor(magicSquare: number[][]) {
		this.magicSquare = magicSquare;
		this.magicSquareDimensions = magicSquare.length;
	}
	
	encrypt(text: string): string {
		let encryptedText = Array(text.length).fill(null);
		
		for (let i = 0; i < text.length; i++) {
			const row = Math.floor(i / this.magicSquareDimensions);
			const col = i % this.magicSquareDimensions;
			const newPos = this.magicSquare[row][col];
			console.log(this.magicSquare, newPos)
			if (newPos < text.length) {
				encryptedText[newPos] = text[i];
			}
		}
		
		return encryptedText.join("");
	}
	
	decrypt(encryptedText: string): string {
		let decryptedText = Array(encryptedText.length).fill(null);
		
		for (let i = 0; i < encryptedText.length; i++) {
			const row = Math.floor(i / this.magicSquareDimensions);
			const col = i % this.magicSquareDimensions;
			const originalPos = this.magicSquare[row][col];
			if (originalPos < encryptedText.length) {
				decryptedText[i] = encryptedText[originalPos];
			}
		}
		
		return decryptedText.join("");
	}
}
\end{lstlisting}

\section{Алгоритм Кузнечик: шифрование/расшифрование данных в режиме простой замены}

\subsection{Задание}

Составить алгоритм программы шифрования по выбранному методу.
Составить программу шифрования по соответствующему заданию. 
Составить алгоритм программы расшифрования по выбранному
методу.
Составить программу расшифрования по соответствующему заданию.

\subsection{Выполнение практической}

Результат практической: \url{https://crypto-tasks.vercel.app/task3}.

Репозиторий \url{https://github.com/ValeryVerkhoturov/crypto-tasks}.

\begin{lstlisting}[caption={Шифр Магический квадрат}, label=lst:2]
let tabl_notlin: Buffer=Buffer.from([
252, 238, 221, 17, 207, 110, 49, 22, 251, 196, 250, 218, 35, 197, 4, 
77, 233, 119, 240, 219, 147, 46, 153, 186, 23, 54, 241, 187, 20, 205, 
95, 193, 249, 24, 101, 90, 226, 92, 239, 33, 129, 28, 60, 66, 139, 1, 
142, 79, 5, 132, 2, 174, 227, 106, 143, 160, 6, 11, 237, 152, 127, 212, 
211, 31, 235, 52, 44, 81, 234, 200, 72, 171, 242, 42, 104, 162, 253, 58, 
206, 204, 181, 112, 14, 86, 8, 12, 118, 18, 191, 114, 19, 71, 156, 183, 
93, 135, 21, 161, 150, 41, 16, 123, 154, 199, 243, 145, 120, 111, 157, 
158, 178, 177, 50, 117, 25, 61, 255, 53, 138, 126, 109, 84, 198, 128, 195, 
189, 13, 87, 223, 245, 36, 169, 62, 168, 67, 201, 215, 121, 214, 246, 124, 
34, 185, 3, 224, 15, 236, 222, 122, 148, 176, 188, 220, 232, 40, 80, 78, 
51, 10, 74, 167, 151, 96, 115, 30, 0, 98, 68, 26, 184, 56, 130, 100, 159, 
38, 65, 173, 69, 70, 146, 39, 94, 85, 47, 140, 163, 165, 125, 105, 213, 
149, 59, 7, 88, 179, 64, 134, 172, 29, 247, 48, 55, 107, 228, 136, 217, 
231, 137, 225, 27, 131, 73, 76, 63, 248, 254, 141, 83, 170, 144, 202, 216, 
133, 97, 32, 113, 103, 164, 45, 43, 9, 91, 203, 155, 37, 208, 190, 229, 
108, 82, 89, 166, 116, 210, 230, 244, 180, 192, 209, 102, 175, 194, 57, 75, 99, 182
])
let constants1:Buffer = Buffer.from([148, 32, 133, 16, 194, 192, 1, 251,
1, 192, 194, 16, 133, 32, 148, 1]);

export class Kuznec{
	C: Buffer[];
	iterKey: Buffer[];
	
	constructor(){
		this.iterKey = [];
		this.C = [];
		this.KeyGen();
	};
	
	GaloisMult(value1:number, value2:number){
		let gm: number = 0;
		let hi_bit: number;
		for(let i = 0; i < 8; i++){
			if(value2 & 1){
				gm ^= value1;
			}
			hi_bit = value1 & 0x80;
			value1 <<= 1;
			if(hi_bit){
				value1 ^= 0xc3;
			}
			value2 >>= 1;
		}
		return gm;
	}
	
	XSL(plaintext: Buffer, j: number){
		plaintext = this.XOR(plaintext, this.iterKey[j]);
		plaintext = this.S(plaintext);
		plaintext = this.L(plaintext);
		return plaintext;
	}
	
	LrSrX(cipherText: Buffer, j: number){
		cipherText = this.L_rev(cipherText);
		cipherText = this.S_rev(cipherText);
		cipherText = this.XOR(cipherText, this.iterKey[j]);
		return cipherText;
	}
	
	Decryption(cipherText : Buffer){
		cipherText = this.XOR(cipherText, this.iterKey[this.iterKey.length - 1]);
		
		for(let i = this.iterKey.length - 2; i >= 0; i--){
			cipherText = this.LrSrX(cipherText, i);
		}
		return cipherText;
	}
	
	XOR(a: Buffer, b: Buffer){
		let result: Buffer = Buffer.alloc(16);
		for(let i = 0; i < 16; i++){
			result[i] = a[i] ^ b[i];
		}
		return result
	}
	
	Encryption(plaintext : Buffer){
		for(let i = 0; i < this.iterKey.length - 1; i++){
			plaintext = this.XSL(plaintext, i);
		}
		plaintext = this.XOR(plaintext, this.iterKey[this.iterKey.length - 1]);
		return plaintext;
	}
	
	ConstGen(){
		this.C=[];
		for(let i = 1; i <= 32; i++){
			let z: number =i;
			let m: Buffer=Buffer.from([0,0,0,0,0,0,0,0,0,0,0,0,0,0,0,0])
			m[15]=z;
			this.C.push(this.L(m));
		}
		return this.C;
	}
	
	GOSTF(key1: Buffer, key2:Buffer, iter_const: Buffer){
		let internal: Buffer = Buffer.alloc(0);
		let outKey2 = key1;
		internal = this.XOR(key1, iter_const);
		internal = this.S(internal);
		internal = this.L(internal);
		
		let outKey1: Buffer = Buffer.alloc(0);
		outKey1 = this.XOR(internal, key2);
		
		let key: Buffer[] = [];
		key.push(outKey1);
		key.push(outKey2);
		return key;
	}
	KeyGen(){
		let temp: number[] = [];
		for(let i = 0; i < 32; ++i){
			temp.push(Math.floor(Math.random() * 255))
		}
		this.GetKeys(Buffer.from(temp));
	}
	GetKeys(masterkey: Buffer){
		let key1: Buffer = Buffer.alloc(16); let key2: Buffer = Buffer.alloc(16);
		for(let i = 0; i < 16; i++){
			key1[i] = masterkey[i];
		}
		for(let i = 16; i < 32; i++){
			key2[i - 16] = masterkey[i];
		}
		let i: number;
		
		let iter12: Buffer[] = [Buffer.alloc(0), Buffer.alloc(0)];
		let iter34: Buffer[] = [Buffer.alloc(0), Buffer.alloc(0)];
		this.ConstGen();
		this.iterKey[0] = key1;
		this.iterKey[1] = key2;
		iter12[0] = key1;
		iter12[1] = key2;
		
		for(i = 0; i < 4; i++){
			for(let j = 0; j < 8; j +=2 ){
				iter34 = this.GOSTF(iter12[0], iter12[1], this.C[j + 8*i]);
				iter12 = this.GOSTF(iter34[0], iter34[1], this.C[j + 1 + 8*i]);
			}
			
			this.iterKey[2 * i + 2] = iter12[0];
			this.iterKey[2 * i + 3] = iter12[1];
		}
		
		return this.iterKey;
	}
	
	GOSTR(bytes: Buffer){
		let r: Buffer = Buffer.from([0, 0, 0, 0, 0, 0, 0, 0, 0, 0, 0, 0, 0, 0, 0, 0]);
		let a15: number = 0;
		for(let i = 15; i >= 1; i--){
			r[i] = bytes[i-1];
		}
		for(let i = 0; i <16; i++){
			a15 ^= this.GaloisMult(constants1[i], bytes[i]);
			
		}
		r[0] = a15;
		return r;
	}
	
	L(bytes: Buffer){
		let result: Buffer = bytes.slice();
		for(let i = 0; i < 16; i++){
			result = this.GOSTR(result);
		}
		return result;
	}
	
	S (bytes: Buffer){
		let result: Buffer = Buffer.alloc(0);
		for(let i:number=0; i<bytes.length;i++){
			result = Buffer.concat([result, Buffer.from([tabl_notlin[bytes[i]]])]);
		}
		while(result.length < 16){
			result = Buffer.concat([Buffer.from([0]), result]);
		}
		return result;
	}
	
	GOSTR_rev(a: Buffer){
		let a_0: number;
		a_0 = 0;
		let r_inv: Buffer = Buffer.from([0, 0, 0, 0, 0, 0, 0, 0, 0, 0,
		0, 0, 0, 0, 0, 0]);
		for(let i = 0; i < 15; i++){
			r_inv[i] = a[i+1];
		}
		a_0 = a[0];
		for (let i = 0; i < 15; i++)
		{
			a_0 ^= this.GaloisMult(a[i + 1], constants1[i]);
		}
		r_inv[15] = a_0;
		return r_inv;
	}
	
	L_rev (bytes: Buffer){
		
		let res: Buffer = bytes.slice();
		
		for(let j = 0; j < 16; j++){
			res = this.GOSTR_rev(res);
		}
		return res;	
	}
	
	S_rev (bytes: Buffer){
		while(bytes.length < 16){
			bytes = Buffer.concat([bytes, Buffer.from([0])]);
		}
		let result: Buffer=Buffer.from([0, 0, 0, 0, 0, 0, 0, 0, 0, 0, 0, 0, 0, 0, 0, 0]);
		for(let i:number=0; i<16;i++){
			result[i] = tabl_notlin.indexOf(bytes[i]);
		}
		return result;
	}
}
export function HexInput(byte:string){
	return(Buffer.from(byte.replace(/\s+/g, ''), 'hex'));
}


export class ECB{
	kuz: Kuznec;
	
	constructor(){
		this.kuz = new Kuznec();
	}
	
	GetKeys(){
		return this.kuz.iterKey;
	}
	
	SetKeys(keys: Buffer[]){
		this.kuz.iterKey = keys;
		return this.kuz.iterKey;
	}
	
	Encrypt(input:Buffer){
		let numbl: number = Math.floor(input.length/16);
		let out: Buffer = Buffer.alloc((numbl+1)*16);
		for(let i:number = 0; i< numbl; i++){
			let temp:Buffer = Buffer.alloc(16);
			temp = this.kuz.Encryption(input.slice(i*16,i*16 + 16))
			for(let j:number = 0; j<16;j++){
				out[16*i+j]=temp[j];
			}
		}
		if(input.length%16!=0){
			let temp:Buffer = Buffer.alloc(16);
			temp = this.kuz.Encryption(input.slice(numbl*16,numbl*16 + input.length%16))
			for(let j:number=0;j<16;j++){
				out[16*numbl+j]=temp[j]; 
			}
		}
		return out;
	}
	
	Decrypt(encrypted:Buffer){
		let decrypted: Buffer = Buffer.alloc(encrypted.length);
		let numbl: number = encrypted.length/16;
		// for(let i = 0; i < encrypted.length; i++){
			//     decrypted.push(this.kuz.Decryption(encrypted[i]));
			// }
		for(let i = 0; i < numbl; i++){
			let temp: Buffer = Buffer.alloc(16);
			temp = this.kuz.Decryption(encrypted.slice(i*16,i*16 + 16))
			for(let j:number = 0; j<16;j++){
				decrypted[16*i+j]=temp[j];
			}
		}
		return decrypted;
	}
}
\end{lstlisting}



\section*{Заключение}
\phantomsection
\addcontentsline{toc}{section}{Заключение}

Были реализованы алгоритмы шифров многоалфавитной замены Вижинера, магический квадрат, алгоритм кузнечик в режиме простой замены.


