% !TeX program = xelatex
% !TeX encoding = UTF-8
% !TeX root = verkhoturov.tex

\begin{titlepage}
	
	\pagestyle{empty}
	\setlength\parindent{0pt}
	\newcommand{\blankDate}[2]{\mbox{\uline{<<\makebox[.7cm]{#1}>>~\makebox[2cm]{#2}~\the\year{}~г.}}} % {день}{месяц}
	\newcommand\blankLine[2]{$\underset{\text{#1}}{\text{\uline{#2}}}$}
	\begin{center}
		\includegraphics[width=2.5cm]{MIREA_Gerb_Black} \par
		МИНОБРНАУКИ РОССИИ \par 
		Федеральное государственное бюджетное образовательное учреждение высшего образования \par
		\textbf{<<МИРЭА~--- Российский технологический университет>>} \par
		\textbf{\fontsize{16pt}{16pt}\selectfont РТУ МИРЭА} \par
		\blankLine{(наименование института, филиала)}{Институт кибербезопасности и цифровых технологий} \par
		\blankLine{(наименование кафедры)}{Кафедра КБ-14 <<Цифровые технологии обработки данных>>} \par
		\vspace*{1cm}
		{\fontsize{16pt}{16pt}\selectfont
			\textbf{Практическая работа}} \par
		по дисциплине \blankLine{(наименование дисциплины)}{Криптографическая защита информации}
	\end{center}
	\vspace*{1cm}
	\begin{flushright}
		Выполнил: \par
		БСБО-05-20 \par
		Верхотуров В. С.
	\end{flushright}
	
	\begin{center}
		\vfill Москва~--- \the\year{}~г.
	\end{center}
\end{titlepage}
\addtocounter{page}{3}


\tableofcontents

\section{Шифр многоалфавитной замены Вижинера}

\subsection{Задание}

Для повышения стойкости шифра используют многоалфавитные замены,
в которых для замены символов исходного текста используются символы
нескольких алфавитов.

Одной из разновидностей такого метода является схема шифрования
Вижинера. Шифр очень устойчивый к вскрытию. Таблица Вижинера
представляет собой квадратную матрицу с n2 элементами, где n – число
символов используемого алфавита. Каждая строка получена циклическим
сдвигом алфавита на один символ

При шифровании сообщения его выписывают в строку, а под ним
буквенный ключ. Если ключ оказался короче сообщения, то его циклически
повторяют. Шифровку получают, находя символ в колонке таблицы по букве
текста и строке, соответствующей букве ключа.

Например:

Сообщение П Р И Е З Ж А Ю Ш Е С Т О Г О

Ключ А Г А В А А Г А В А А Г А В А

Шифровка П Н И Г З Ж Ю Ю ЮА Е О Т М

Предыдущие шифры называются монограммными, так как шифрование ведется по одной букве. Шифрование по 2 букве называются биграммными.

\subsection{Выполнение практической}

Результат практической: \url{https://crypto-tasks.vercel.app/task1}.

Репозиторий \url{https://github.com/ValeryVerkhoturov/crypto-tasks}.

\begin{lstlisting}[caption={Шифр многоалфавитной замены Вижинера}, label=lst:1]
export class VigenereCipher {
	private alphabet: string = 'АБВГДЕЁЖЗИЙКЛМНОПРСТУФХЦЧШЩЪЫЬЭЮЯ';
	private mod: number = this.alphabet.length;
	
	encrypt(text: string, key: string): string {
		return this.processText(text, key, 'encrypt');
	}
	
	decrypt(text: string, key: string): string {
		return this.processText(text, key, 'decrypt');
	}
	
	private processText(text: string, key: string, mode: 'encrypt' | 'decrypt'): string {
		let processedText = '';
		let keyIndex = 0;
		
		text = text.toUpperCase();
		key = key.toUpperCase();
		
		for (let i = 0; i < text.length; i++) {
			const char = text[i]
			if (this.alphabet.includes(char)) {
				const charIndex = this.alphabet.indexOf(char);
				const keyChar = key[keyIndex % key.length];
				const keyCharIndex = this.alphabet.indexOf(keyChar);
				
				if (mode === 'encrypt') {
					processedText += this.alphabet[(charIndex + keyCharIndex) % this.mod];
				} else {
					let decodeIndex = (charIndex - keyCharIndex) % this.mod;
					if (decodeIndex < 0) {
						decodeIndex += this.mod;
					}
					processedText += this.alphabet[decodeIndex];
				}
				
				keyIndex++;
			} else {
				processedText += char;
			}
		}
		
		return processedText;
	}
}

\end{lstlisting}

\section{Магический квадрат}

\subsection{Задание}

Магический квадрат - это квадратная таблица с вписанными в клетки последовательными натуральными числами от 1, которые дают в сумме по каждому столбцу, каждой строке и каждой диагонали одно и тоже число.

Чтобы зашифровать открытый текст с помощью такого квадрата, нужно пронумеровать все буквы в фразе по порядку и вставить их в квадрат вместо соответствующих цифр.

\subsection{Выполнение практической}

Результат практической: \url{https://crypto-tasks.vercel.app/task2}.

Репозиторий \url{https://github.com/ValeryVerkhoturov/crypto-tasks}.

\begin{lstlisting}[caption={Шифр Магический квадрат}, label=lst:2]
export class MagicSquareCipher {
	private magicSquare: number[][]
	private magicSquareDimensions: number
	
	constructor(magicSquare: number[][]) {
		this.magicSquare = magicSquare;
		this.magicSquareDimensions = magicSquare.length;
	}
	
	encrypt(text: string): string {
		let encryptedText = Array(text.length).fill(null);
		
		for (let i = 0; i < text.length; i++) {
			const row = Math.floor(i / this.magicSquareDimensions);
			const col = i % this.magicSquareDimensions;
			const newPos = this.magicSquare[row][col];
			console.log(this.magicSquare, newPos)
			if (newPos < text.length) {
				encryptedText[newPos] = text[i];
			}
		}
		
		return encryptedText.join("");
	}
	
	decrypt(encryptedText: string): string {
		let decryptedText = Array(encryptedText.length).fill(null);
		
		for (let i = 0; i < encryptedText.length; i++) {
			const row = Math.floor(i / this.magicSquareDimensions);
			const col = i % this.magicSquareDimensions;
			const originalPos = this.magicSquare[row][col];
			if (originalPos < encryptedText.length) {
				decryptedText[i] = encryptedText[originalPos];
			}
		}
		
		return decryptedText.join("");
	}
}
\end{lstlisting}




%\begin{figure}[htb]
%	\centering
%	\includegraphics[width=.9\textwidth]{graph.png}
%	\parskip=6pt
%	\caption{Визуализация создаваемого графа}
%	\label{fig:gfpic}
%\end{figure}



\section*{Заключение}
\phantomsection
\addcontentsline{toc}{section}{Заключение}


