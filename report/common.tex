% !TeX program = xelatex
% !TeX encoding = UTF-8
% !TeX root = verkhoturov.tex

\begin{titlepage}
	
	\pagestyle{empty}
	\setlength\parindent{0pt}
	\newcommand{\blankDate}[2]{\mbox{\uline{<<\makebox[.7cm]{#1}>>~\makebox[2cm]{#2}~\the\year{}~г.}}} % {день}{месяц}
	\newcommand\blankLine[2]{$\underset{\text{#1}}{\text{\uline{#2}}}$}
	\begin{center}
		\includegraphics[width=2.5cm]{MIREA_Gerb_Black} \par
		МИНОБРНАУКИ РОССИИ \par 
		Федеральное государственное бюджетное образовательное учреждение высшего образования \par
		\textbf{<<МИРЭА~--- Российский технологический университет>>} \par
		\textbf{\fontsize{16pt}{16pt}\selectfont РТУ МИРЭА} \par
		\blankLine{(наименование института, филиала)}{Институт кибербезопасности и цифровых технологий} \par
		\blankLine{(наименование кафедры)}{Кафедра КБ-14 <<Цифровые технологии обработки данных>>} \par
		\vspace*{1cm}
		{\fontsize{16pt}{16pt}\selectfont
			\textbf{Практическая работа}} \par
		по дисциплине \blankLine{(наименование дисциплины)}{Криптографическая защита информации}
	\end{center}
	\vspace*{1cm}
	\begin{flushright}
		Выполнил: \par
		БСБО-05-20 \par
		Верхотуров В. С.
	\end{flushright}
	
	\begin{center}
		\vfill Москва~--- \the\year{}~г.
	\end{center}
\end{titlepage}
\addtocounter{page}{3}


\tableofcontents

\section{Задание 1. Шифр биграммами}

\subsection{Задание}

Наиболее известный шифр биграммами называется Playfair
(использовался в I Мировой войне). Открытый текст разбивается на пары
(биграммы). Текст шифруется по следующим правилам:

--- Если обе буквы биграммы исходного текста принадлежат одному
столбцу таблицы, то буквами шифра считаются буквы, которые лежат под
ними. Если буква открытого текста находится в нижнем ряду, то для шифра
берется соответствующая буква из верхнего ряда. Биграмма из одной буквы
или пары одинаковых букв тоже подчиняются этому правилу.

--- Если обе буквы биграммы исходного текста принадлежат одной строке
таблицы, то буквами шифра считаются буквы, которые лежат справа от них.
Если буква открытого текста находится в правой колонке (в последней), то для
шифра берется соответствующая буква из первой колонки.

--- Если обе буквы биграммы открытого текста лежат в разных столбцах,
то вместо них берутся такие 2 буквы, чтобы вся четверка их представляла
прямоугольник. При этом последовательность букв в шифре должна быть
зеркальной исходной паре.

Пример для таблицы 5×6 с ключом «Республика» 

Открытый текст ПУ СТ ЬК ОН СУ ЛЫ БУ ДУ ТБ ДИ ТЕ ЛЬ НЫ

Шифр УБ РХ СЗ ДО ПБ ИЩ РБ НР ШР ЖЛ ФР КЩ ЗЮ

\subsection{Выполнение практической}

Результат практической: \url{https://crypto-tasks.vercel.app/task1}.

Репозиторий \url{https://github.com/ValeryVerkhoturov/crypto-tasks}.

\begin{lstlisting}[caption={Playfair шифр}, label=lst:gf1]
export class PlayfairCipher {
	private readonly key;
	private gridSize: number = 6;
	private grid: string[][] = [];
	private positionMap: Map<string, {row: number, col: number}> = new Map();
	
	constructor(key: string) {
		this.key = key;
		this.initializeGrid();
	}
	
	private initializeGrid(): void {
		const alphabet = "АБВГДЕЁЖЗИЙКЛМНОПРСТУФХЦЧШЩЪЫЬЭЮЯ";
		let keyString = Array.from(new Set(this.key.replace(/\s+/g, '').toUpperCase() + alphabet))
		.filter(char => alphabet.includes(char)).join('');
		
		for (let i = 0; i < this.gridSize; i++) {
			this.grid[i] = [];
			for (let j = 0; j < this.gridSize; j++) {
				const char = keyString[(i * this.gridSize) + j];
				this.grid[i][j] = char;
				this.positionMap.set(char, { row: i, col: j });
			}
		}
	}
	
	encrypt(plaintext: string): string {
		return this.processText(plaintext, 'encrypt');
	}
	
	decrypt(ciphertext: string): string {
		return this.processText(ciphertext, 'decrypt');
	}
	
	private processText(inputText: string, mode: 'encrypt' | 'decrypt'): string {
		let processedText = inputText.toUpperCase().replace(/[^А-ЯЁ]+/g, "");
		let output = "";
		
		for (let i = 0; i < processedText.length; i += 2) {
			if (i + 1 >= processedText.length) {
				processedText += "Ь";
			}
			
			if (processedText[i] === processedText[i + 1]) {
				processedText = processedText.substring(0, i + 1) + "Ь" + processedText.substring(i + 1);
			}
			
			const pos1 = this.positionMap.get(processedText[i]);
			const pos2 = this.positionMap.get(processedText[i + 1]);
			
			if (pos1 && pos2) {
				let row1 = pos1.row;
				let col1 = pos1.col;
				let row2 = pos2.row;
				let col2 = pos2.col;
				
				if (row1 === row2) {
					col1 = this.mod((col1 + (mode === 'encrypt' ? 1 : -1)), this.gridSize);
					col2 = this.mod((col2 + (mode === 'encrypt' ? 1 : -1)), this.gridSize);
				} else if (col1 === col2) {
					row1 = this.mod((row1 + (mode === 'encrypt' ? 1 : -1)), this.gridSize);
					row2 = this.mod((row2 + (mode === 'encrypt' ? 1 : -1)), this.gridSize);
				} else {
					[col1, col2] = [col2, col1];
				}
				
				output += this.grid[row1][col1] + this.grid[row2][col2];
			}
		}
		
		return output;
	}
	
	private mod(n: number, m: number): number {
		return ((n % m) + m) % m;
	}
}
\end{lstlisting}



%\begin{figure}[htb]
%	\centering
%	\includegraphics[width=.9\textwidth]{graph.png}
%	\parskip=6pt
%	\caption{Визуализация создаваемого графа}
%	\label{fig:gfpic}
%\end{figure}



\section*{Заключение}
\phantomsection
\addcontentsline{toc}{section}{Заключение}


